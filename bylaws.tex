\documentclass[a4paper]{article}
\usepackage[utf8]{inputenc}

% ==============================================================================
% Variables:
% ==============================================================================

\newcommand{\name}{Chalmers Software Craftsmanship Guild}
\newcommand{\shortname}{CSCG}

% ==============================================================================
% Settings:
% ==============================================================================

\usepackage[T1]{fontenc}
\usepackage[swedish]{babel}
\usepackage[iso,swedish]{isodate}
\usepackage{icomma}
\usepackage{graphicx,tabu}
\usepackage{todonotes}
\usepackage{changepage}
\usepackage[hidelinks]{hyperref}

\usepackage[
    %showframe,
    a4paper,
    %left=2.5cm,
    %right=2cm,
    top=1cm,
    bottom=2.5cm
    ]{geometry}
\setlength\headheight{2.5cm}
\setlength\textheight{24cm}

\parskip .5em
\parindent 0em

\usepackage{lastpage}
\usepackage{fancyhdr}
\pagestyle{fancy}

\lhead{}%\includegraphics[height=1.5cm]{logo}}
\rhead{\thepage\ (\pageref{LastPage})}
\chead{{\large Bylaws\\ \name}}

\renewcommand{\thesection}{§~\arabic{section}}

\author{}
\date{}
\title{\huge Bylaws for\\
    \huge \name}

\begin{document}

% ==============================================================================
% begin: document
% ==============================================================================

\maketitle

\begin{center}
    \textbf{Revision history}\\
    \begin{tabu}{ll}
        Accepted at constitutional meeting  & 2017-05-12
    \end{tabu}
\end{center}


% =========================================================================== %

\section{Society}

\name{} is a non-profit society which is religiously and politically impartial.

\name{} is a society under ``Chalmers Studentkår'', the Chalmers Student Union,
and is regulated by the bylaws and rules determined by Chalmers Studentkår.

\section{Vision}
\label{sec:vision}

\name{} exists to promote the interest, knowledge and ethical application of
cyber security, with a practical focus rooted in playing and holding Capture
the flag events. The guild should also promote and encourage free and open
source software using a software craftsmanship approach.

% =========================================================================== %

\section{Membership}

Membership in \name{} is available to anyone, but at least 50 percent of the
society's members must be members or supporting members of Chalmers Student
Union. The membership fee is decided on a fiscal yearly basis by the annual
meeting.

\subsection{Validity}

After a person has expressed a wish to join \name{}, they are considered
a member of the society  when they have been added to the member database.
A membership is valid until the end of the fiscal year.

\subsection{Expulsion}

A member who actively opposes the vision of \name{} can be expelled from the
society. A decision regarding expulsion of a member must be made by the annual
meeting with at least two thirds majority.

% =========================================================================== %

\section{Economy}

The economy of the society is reported by the treasurer at the annual meeting,
for the previous fiscal year.  The society's accountants review the society's
accounts and write an audit report to be presented at the annual meeting for the
same fiscal year.

All material requested for the audit by the Student Union shall be available two
months after the end of the fiscal year or two weeks before the audit of the
Chalmers Student Union, which ever comes first.

% =========================================================================== %

\section{Organization}

\subsection{Fiscal- and financial year}

The society's fiscal- and financial year comprises the 1st of July until the
30th of June.

\subsection{Board}

\name{}'s board consists of one chairman, one treasurer and between one and five
members.  The board is elected by the annual meeting, for the following fiscal
year. The board leads the society's activities and manages its possessions.

\subsubsection{Responsibilities}

It is the responsibility of the chairman to lead the board
and it is the responsibility of the treasurer to manage the economy of the
society.

\subsubsection{Decision making}

Any decisions made by the board shall be made with an absolute majority. In the
case of a tie, the chairman has the deciding vote.

\subsubsection{Validity of decisions}

The board is deemed to have a quorum if and only if all
members of the board have been summoned, and at least half of the board's
members are present.

\subsubsection{Yearly report}

The board is responsible for writing a yearly report where the
activities of the past year are presented. The board is also responsible for
ensuring the succeeding board receives said report.

% =========================================================================== %

\subsection{Annual meeting}
\label{sec:annualmeeting}

The society is governed by the annual meeting. Between annual meetings the
society is lead by the board. An annual meeting is to be held at the end of each
fiscal year.

\subsubsection{Motions}

Any motions that are to be regarded at the meeting must be present in the
summoning.

\newpage
\subsubsection{Agenda}
\label{sec:agenda}

The annual meeting must discuss the following:

\begin{enumerate}
    \item The opening of the meeting
    \item Elect chairman for the meeting
    \item Elect secretary for the meeting
    \item Elect two adjusters whom also serve as vote counters
    \item Approval of the agenda
    \item Ask if the summoning of the meeting was made in accordance to the bylaws
    \item Set the membership fee
    \item Present the yearly report
    \item Present the audit report
    \item Decide if the board of the previous fiscal year is to be discharged
    \item Treat any propositions
    \item Treat any motions
    \item Elect chairman for the coming year
    \item Elect treasurer for the coming year
    \item Elect board members for the coming year
    \item Elect accountants for the coming year
    \item Elect election committee for the coming year
    \item Other questions
\end{enumerate}

\subsubsection{Validity}

The annual meeting is able to make decisions if and only if a
written summoning has been sent out via the currently most used communication
channels to all members at least two weeks before the meeting.

\subsubsection{Voting}

All members of \name{} are allowed to vote at annual meetings.
The vote is personal, but can be delegated to one member that is present at the
meeting. The process of delegating votes requires a written approval from the
person whose vote is delegated.

\subsubsection{Decision making}
\label{sec:annualdecision}

All decisions at the annual meeting are made by acclamation or
by simple majority in a vote. This is however not applicable if the meeting wishes to
dissolve the society or change the bylaws (see \ref{sec:dissolve} and \ref{sec:changingbylaws} respectively).

\subsubsection{Entitled attendance}

The chairman, the board, the accountants and the members of
the society are all entitled to be present and make their voices heard during
the annual meeting.

\subsubsection{Personal elections}

Any personal elections shall be conducted using open voting,
unless a closed voting is requested by any member. If there is only
one person suggested for a particular election, the vote is made in accordance
to \ref{sec:annualdecision}.

\subsubsection{Tie break}

In the case of a tie, the chairman holds the final vote
unless in the case of a personal election where a toss is used.

\subsection{Extra annual meeting}

An extra annual meeting is summoned to on the initiative of the
board or if at least 25~percent of all members requests so.

The agenda of the extra annual meeting must discuss points 1--6
in~\ref{sec:agenda}. The summoning must state the reason for calling the extra
annual meeting.

% =========================================================================== %

\subsection{The society's firm}

The society's firm is singly signed by the chairman and the
treasurer.

% =========================================================================== %

\subsection{Changing bylaws}
\label{sec:changingbylaws}

These bylaws can be changed only by an annual meeting or an extra annual
meeting. If the meeting intends to change the bylaws, this must have been
clearly stated in the summoning to the meeting.

In order for the bylaws to be changed, any suggested changes must be supported by
at least two thirds of the voters present at the meeting.

\subsubsection{Major changes}

For changes of the bylaws concerning \ref{sec:vision} (\nameref{sec:vision}),
\ref{sec:changingbylaws} (\nameref{sec:changingbylaws}) or \ref{sec:dissolve}
(\nameref{sec:dissolve}), two consecutive annual meetings must support the
change. In order for an extra annual meeting to approve said changes, it needs
to be held at least one month apart from the ordinary annual meeting that also
intends to approve the changes.

% =========================================================================== %

\section{Dissolving the society}
\label{sec:dissolve}

The society may only be dissolved by two consecutive annual meetings, which are
held at least one month apart.  Two thirds of the present voters of said
meetings must support the dissolution.  The matter of dissolving the society
must be clearly stated in the summoning to the meetings.

\subsection{Granting of assets}

If the society is dissolved any financial means, or possessions of the society,
is donated to the Student Union, with the wish to be used towards
fulfilling~\ref{sec:vision}~\nameref{sec:vision} as it is stated when the
dissolution takes effect.

\end{document}


